\subsection*{Ejercicio 2.1}
En el problema de la braqu\'istocrona, se llega a que se debe minimizar el funcional:
\eq{
\label{integ}
S&=\int_1^2 \sqrt{\frac{1+y'(x)^2}{2g\,y(x)}}\,dx,\\
f&=\sqrt{\frac{1+y'(x)^2}{2g\,y(x)}}.
}
La ecuaci\'on de Euler-Lagrange para el problema se obtiene f\'acilmente haciendo los siguientes c\'alculos:
\eq{
\frac{\partial f}{\partial y'(x)}&=\frac{y'(x)}{\sqrt{(2 g y(x))(1+y'(x)^2)}},\\
\frac{\partial f}{\partial y(x)}&=\frac{\sqrt{1+y'(x)^2}}{2 y \sqrt{2 g y}},\\
0&=\frac{d}{dx}\left(\frac{\partial f}{\partial y'(x)} \right)-\frac{\partial f}{\partial y(x)}=\frac{d}{dx}\left( \frac{y'(x)}{\sqrt{(2 g y(x))(1+y'(x)^2)}}\right)-\frac{\sqrt{1+y'(x)^2}}{2 y \sqrt{2 g y}}\nonumber\\
&=-\frac{y'(x)^2+2 y(x) y''(x)+1}{2 \sqrt{2} g^2 y(x)^3\left(\frac{y'(x)^2+1}{g y(x)}\right)^{3/2}}.
}
La ecuaci\'on diferencial que debemos resolver es entonces:
\eq{
y'(x)^2+2 y(x) y''(x)+1=0.
\label{n}
}
Podemos notar que $f$ no depende expl\'icitamente de $x$, entonces la cantidad $h=\frac{\partial f}{\partial y'(x)}y'(x)-f$ es una cantidad conservada. Para este caso $h$ tiene la siguiente forma:
\eq{
h&=-\frac{1}{\sqrt{2 g y(x)(1+y'(x)^2)}}=\text{cte},\\
\frac{1}{h^2}&=2 g y(x)(1+y'(x)^2).
}
Si se deriva la \'ultima de las ecuaciones con respecto a $x$ se llega a:
\eq{
0=\frac{d}{dx}\left(\frac{1}{h^2}\right)=g y'(x)\left(y'(x)^2+1\right)+2 g y(x) y'(x) y''(x),
}
que es la misma ecuaci\'on (\ref{n}) multiplicada por $g y'(x)$. Es decir, la ecuaci\'on (\ref{n}) es equivalente a:
\eq{
y(x)(1+y'(x)^2)=2c.
}
De esta ecuaci\'on se puede despejar $y'(x)$, separar variables e integrar para obtener:
\eq{
y'(x)^2&=\frac{2c}{y}-1,\\
\frac{dy}{\sqrt{\frac{2c}{y}-1}}&=dx,\\
2 c \tan^{-1}\left(\frac{1}{\sqrt{\frac{2c}{y}-1}}\right)-\sqrt{(2 c-y)y}&=x.
}
Pero $\tan^{-1}(x)=\cos^{-1}\left(\frac{1}{\sqrt{1+x^2}}\right)$, entonces:
\eq{
2 c \cos^{-1}\left(\sqrt{1-\frac{y}{2c}}\right)-\sqrt{(2 c-y)y}=x,\\
1-\frac{y}{2c}=\cos^2\left(\frac{x+\sqrt{y(2c-y)}}{2c}\right).
}
Llamando $\alpha=\frac{x+\sqrt{y(2c-y)}}{c}$ y despejando:
\eq{
\frac{y}{2c}=1-\cos^2\left(\frac{\alpha}{2}\right)=\frac{1-\cos(\alpha)}{2},\\
\frac{y}{c}=1-\cos\left(\frac{x+\sqrt{y(2c-y)}}{c}\right).
}
La \'ultima igualdad se obtiene de las identidades de \'angulo mitad para el coseno.
De esta igualdad tambi\'en se ve que $y$ s\'olo puede tomar valores en el intervalo $[0,2a]$ (ya que la cantidad dentro del radical del coseno s\'olo es mayor que 0 en este intervalo) y que uno de sus m\'inimos est\'a en $(0,0)$ (que fue de donde se lanz\'o la part\'icula), es decir, all\'i hay una c\'uspide.\\
Por otro lado, para mostrar que si la part\'icula se proyecta con una energ\'ia cin\'etica inicial $\frac{1}{2}m v_0^2$ entonces tambi\'en se mueve sobre un cicloide, basta notar que la integral (\ref{integ}) se reescribe de esta forma (conservaci\'on de energ\'ia: $\frac{1}{2}m v_0^2=\frac{1}{2} mv^2-mgy$):
\eq{
S=\int_1^2 \sqrt{\frac{1+y'(x)^2}{2 g y(x)+v_0^2}}\,dx.
}
Esta puede ser devuelta a su forma original haciendo el cambio de variable $\hat y=y+\frac{v_0^2}{2g}$, esto equivale a mover el sistema de coordenadas $\frac{v_0^2}{2g}$ unidades, por lo tanto la c\'uspide que estaba en $y=0$ quedar\'a $\frac{v_0^2}{2g}$ unidades m\'as arriba.

\subsection*{Ejercicio 2.3}
La soluci\'on a este problema consiste en minimizar el funcional:
\eq{
J=\int_1^2 \sqrt{dx^2+dy_1^2+dy_2^2}=\int_{x_1}^{x_2} \sqrt{1+\left(\frac{dy_1}{dx}\right)^2+\left(\frac{dy_2}{dx}\right)^2}\,dx,
}
para esto resolvemos:
\eq{
\label{eu}
0&=\frac{\partial f}{\partial y_i}-\frac{d}{dx}\left(\frac{\partial f}{\partial \dot y_i}\right),\\
f&=\sqrt{1+\dot y_1^2+\dot y_2^2},\\
\dot y_i&=\frac{dy_i}{dx}.
}
Con $f$ as\'i definida obtenemos:
\eq{
\frac{\partial f}{\partial y_i}&=0,\\
\frac{\partial f}{\partial \dot y_i}&=\frac{\dot y_i}{\sqrt{1+\sum_i\dot y_i^2}}.
}
As\'i llegamos a que:
\eq{
\frac{d}{dx}\left(\frac{\dot y_i}{\sqrt{1+\sum_i\dot y_i^2}}\right)=0,\\
\frac{\dot y_i}{\sqrt{1+\sum_i\dot y_i^2}}=c_i.
}
La \'ultima ecuaci\'on se puede solucionar para las $y_i$:
\eq{
\dot y_i^2=\frac{c_i^2}{1-\sum_i c_i^2},
}
y finalmente:
\eq{
y_i=a_i x+b_i.
}
Las anteriores son las ecuaciones de una recta parametrizadas por la primera de sus coordenadas.

\subsection*{Ejercicio 2.4}
Con un procedimiento an\'alogo al anterior pero usando coordenadas esf\'ericas $(r,\theta,\phi)$ (con $r$ fijo) llegamos a que $ds^2=r^2(d\theta^2+\sin^2\theta\,d\phi^2)$, llegamos al siguiente funcional:
\eq{
F=\int_1^2 \sqrt{d\theta^2+\sin^2\theta\,d\phi^2}=\int_{\theta_1}^{\theta_2}\sqrt{1+\sin^2\theta\left(\frac{d\phi}{d\theta}\right)^2}\,d\theta.
}
Haciendo uso de (\ref{eu}) pero con $f=\sqrt{1+\sin^2\theta\left(\frac{d\phi}{d\theta}\right)^2}$, $y_i=\phi$, $x=\theta$ y $\dot\phi=\frac{d\phi}{d\theta}$ obtenemos lo siguiente:
\eq{
\frac{\partial f}{\partial \phi}&=0,\\
\frac{\partial f}{\partial \dot\phi}&=\frac{\sin^2\theta\,\dot\phi}{\sqrt{1+\sin^2\theta\,\dot\phi^2}},\\
0&=\frac{d}{d\theta}\left(\frac{\sin^2\theta\,\dot\phi}{\sqrt{1+\sin^2\theta\,\dot\phi^2}}\right).
}
De la \'ultima ecuaci\'on deducimos que:
\eq{
\frac{\sin^2\theta\,\dot\phi}{\sqrt{1+\sin^2\theta\,\dot\phi^2}}=C_1.
}
De esta y la anterior podemos despejar $\dot\phi$ e integrar para obtener:
\eq{
\dot\phi&=\pm\frac{C_1\csc(\theta)}{\sqrt{\sin^2(\theta)-C_1^2}},\\
\label{pi}
\phi(\theta)&=C_2\pm\tan^{-1}\left(\frac{\sqrt{2}C_1\cos(\theta)}{\sqrt{-2C_1^2-\cos(2\theta)+1}}\right)=C_2\pm\tan^{-1}\left(\frac{\cos\theta}{\sqrt{\frac{\sin^2\theta}{C_1^2}-1}}\right).
}
Ahora nos falta mostrar que la anterior ecuaci\'on define un c\'irculo m\'aximo. Para esto es suficiente con mostrar que el plano en el que est\'a la curva que hemos encontrado anteriormente pasa por el origen.
Para lo anterior, note que $\tan^{-1}(x)=\sin^{-1}\left(\frac{x}{\sqrt{1+x^2}}\right)$ y por lo tanto la ecuaci\'on (\ref{pi}) se puede reescribir as\'i:
\eq{
\phi-C_2=\tan^{-1}\left(\frac{\cot\theta}{\sqrt{\frac{1}{C_1^2}-\csc^2\theta}}\right)=\sin^{-1}\left(\frac{\frac{\cot\theta}{\sqrt{1/C_1^2-\csc^2\theta}}}{\sqrt{1+\frac{\cot^2\theta}{1/C_1^2-\csc^2\theta}}}\right),\\
\sin^{-1}\left(\frac{\cot\theta}{\sqrt{1/C_1^2+\cot^2\theta-\csc^2\theta}}\right)=\sin^{-1}\left(\frac{\cot\theta}{\sqrt{1/C_1^2-1}}\right).
}
O:
\eq{
\sin(\phi-C_2)&=\frac{\cot\theta}{\sqrt{1/C_1^2-1}},\\
\sin\phi\cos C_2-\sin C_2\cos\phi&=\frac{\cot\theta}{\sqrt{1/C_1^2-1}},\\
\sin\theta\sin\phi\cos C_2-\sin\theta\sin C_2\cos\phi&=\frac{\cos\theta}{\sqrt{1/C_1^2-1}}.
}
Pero $x=r\cos\phi\sin\theta$, $y=r\sin\phi\sin\theta$, $z=r\cos\theta$ y la \'ultima ecuaci\'on queda:
\eq{
y\cos C_2-x\sin C_2=\frac{z}{\sqrt{1/C_1^2-1}},
}
que define un plano que pasa por $(0,0,0)$. Note que en la ecuaci\'on anterior $x,y$ y $z$ son funciones de $\theta$, por lo que describen una curva y no un plano.



\subsection*{Ejercicio 2.6}
Del problema que estamos considerando se nota f\'acilmente que la fuerza que atrae a la part\'icula es proporcional a la distancia y est\'a en la direcci\'on radial, por lo tanto el potencial asociado a dicha fuerza es de la forma $V(r)=\frac{1}{2} k r^2$. Escogeremos nuestro sistema de referencia con el origen en el centro de la esfera y con el eje $z$ perpendicular al plano donde se har\'a el t\'unel. As\'i entonces llegamos a que la energ\'ia de la part\'icula que estamos considerando viene dada por:
\eq{
E=\frac{1}{2}m (\dot r^2+r^2 \dot \theta^2)+\frac{1}{2} k r^2=\frac{1}{2}k a^2.
}
Donde $a$ es el radio de la esfera en la que se mover\'a la part\'icula, $\dot w$ denota derivada total con respecto a $t$ mientras $w'$ denota derivada con respecto a $\theta$ y $k=\frac{G M m}{a^3}$. De la anterior ecuaci\'on se puede despejar f\'acilmente ``$dt$'' para obtener:
\eq{
\int dt=\int f\,d\theta=\sqrt{\frac{m}{k}}\int \sqrt{\frac{r(\theta)^2+r'(\theta)^2}{a^2 - r(\theta)^2}}\,d\theta.
}
Al escribir las ecuaciones de Euler-Lagrange para minimizar $\int dt$ se obtiene:
\eq{
\frac{r(\theta)\sqrt{\frac{m\left(r(\theta)^2+r'(\theta)^2\right)}{k\left(a^2-r(\theta)^2\right)}}\left(r''(\theta)r(\theta)^3+\left(a^2-r'(\theta)^2\right)r(\theta)^2-a^2r''(\theta)r(\theta)+2a^2r'(\theta)^2\right)}{r(\theta)^2+r'(\theta)^2}=0.
}
De la anterior igualdad la ecuaci\'on relevante es:
\eq{
r''(\theta)r(\theta)^3+\left(a^2-r'(\theta)^2\right)r(\theta)^2-a^2r''(\theta)r(\theta)+2a^2r'(\theta)^2=0.
}
Aunque la anterior ecuaci\'on se ve algo complicada, esta se puede integrar una vez al notar que $f$ no depende de $\theta$ y por lo tanto:
\eq{
\sqrt{I}=\frac{\partial f}{\partial r'(\theta)}r'(\theta)-f=\frac{r(\theta)^2\sqrt{\frac{m\left(r(\theta)^2+r'(\theta)^2\right)}{k\left(a^2-r(\theta)^2\right)}}}{r(\theta)^2+r'(\theta)^2}.
}
Es una cantidad conservada. La \'ultima igualdad se puede escribir de manera m\'as conveniente como:
\eq{
\frac{a^2r^2}{a^2-r^2}\left(\frac{r^2}{Ia^2}-1+\frac{r^2}{a^2}\right)=r'(\theta)^2.
}
Si redefinimos a $\frac{1}{I}=\left(-1+\frac{a^2}{r_0^2}\right)$ entonces la ecuaci\'on anterior queda as\'i:
\eq{
\frac{a^2r^2}{a^2-r^2}\frac{r^2-r_0^2}{r_0^2}=r'(\theta)^2.
}
Note que la ecuaci\'on anterior restringe los posibles valores $r$ a $r_0\leq r \leq a$ ya que la derivada de $r$ debe ser real.
De la anterior ecuaci\'on basta mostrar que es satisfecha por una hipocicloide.

\begin{figure}
\centering
\includegraphics[width=0.9\textwidth]{dib.eps}
\caption{Construcci\'on de una hipocicloide: Sea $a$ el radio del c\'irculo mayor y $b$ el radio del c\'irculo menor. Sea $\theta$ el \'angulo que forma el vector de posici\'on del punto sobre el c\'irculo menor con el eje $x$, sea $\phi$ el \'angulo que forma el centro del c\'irculo menor y el eje $x$, finalmente sea $\nu$ el \'angulo que rota el c\'irculo menor sobre un eje fijo. De la condici\'on de no deslizamiento se obtiene que $\nu=\frac{a-b}{b}\phi$. Podemos escribir la posici\'on del punto en t\'erminos de $\phi$ y $\nu$ as\'i (definimos a $r_0=a-2b$, la distancia m\'as cercana al centro):\\
$x=(a-b)\cos\phi-b\cos\nu=\frac{1}{2}\left((a+r_0)\cos(\phi)+(r_0-a)\cos\left(\frac{(a+r_0)\phi}{a-r_0}\right)\right)$,\\
$y=(a-b)\sin\phi+b\sin\nu=\frac{1}{2}\left((a+r_0)\sin(\phi)+(a-r_0)\sin\left(\frac{(a+r_0)\phi}{a-r_0}\right)\right)$.}
\label{a}
\end{figure}
Las ecuaciones par\'am\'etricas de una hipocicloide son:
\eq{
x(\phi)&=\frac{1}{2}\left((a+r_{0}) \cos (\phi)+(r_{0}-a) \cos \left(\frac{(a+r_{0}) \phi}{a-r_{0}}\right)\right),\\
y(\phi)&=\frac{1}{2}\left((a+r_{0}) \sin (\phi)+(a-r_{0}) \sin \left(\frac{(a+r_{0}) \phi}{a-r_{0}}\right)\right).
}
De estas se puede obtener $r(\phi)^2$ as\'i:
\eq{
r(\phi)^2=x(\phi)^2+y(\phi)^2&=\frac{1}{4}\left((a+r_0)\cos(\phi)+(r_{0}-a)\cos\left(\frac{(a+r_{0})\phi}{a-r_{0}}\right)\right)^2\\
&+\frac{1}{4}\left((a+r_{0})\sin(\phi)+(a-r_{0})\sin\left(\frac{(a+r_{0})\phi}{a-r_{0}}\right)\right)^2.\nonumber
}
Que al simplificar, teniendo en cuenta que $\cos^2x+\sin^2x=1$, $\cos(a+b)=\cos a\cos b-\sin a\sin b$ y que $(a+b)^2+(a-b)^2=2(a^2+b^2)$, queda:
\eq{
r(\phi)^2=\frac{1}{2}\left(a^2+r_{0}^2+\left(r_{0}^2-a^2\right)\cos\left(\frac{2a\phi}{a-r_{0}}\right)\right).
}
Por otro lado, tambi\'en podemos calcular $\theta(\phi)$:
\eq{
\tan(\theta(\phi))&=\frac{y(\phi)}{x(\phi)}=\frac{(a+r_{0})\sin(\phi)+(a-r_{0})\sin\left(\frac{(a+r_{0})\phi}{a-r_{0}}\right)}{(a+r_{0})\cos(\phi)+(r_{0}-a)\cos\left(\frac{(a+r_{0})\phi}{a-r_{0}}\right)}.
}
Aplicando identidades trigonom\'etricas de suma a producto, se obtiene:
\eq{
\tan(\theta(\phi))&=\frac{a\cos\left(\frac{r_{0}\phi}{a-r_{0}}\right)\sin\left(\frac{a\phi}{a-r_{0}}\right)-r_{0}\cos\left(\frac{a\phi}{a-r_{0}}\right)\sin\left(\frac{r_{0}\phi}{a-r_{0}}\right)}{r_{0}\cos\left(\frac{a\phi}{a-r_{0}}\right)\cos\left(\frac{r_{0}\phi}{a-r_{0}}\right)+a\sin\left(\frac{a\phi}{a-r_{0}}\right)\sin\left(\frac{r_{0}\phi}{a-r_{0}}\right)}.
}
Si dividimos numerador y denominador por $\cos\left(\frac{a\phi}{a-r_{0}}\right)\cos\left(\frac{r_{0}\phi}{a-r_{0}}\right)$, obtenemos:
\eq{
\tan\theta(\phi)=\frac{a\tan\left(\frac{a\phi}{a-r_0}\right)-r_0\tan\left(\frac{r_0\phi}{a-r_0}\right)}{r_0+a\tan\left(\frac{a\phi}{a-r_0}\right)\tan\left(\frac{r_0\phi}{a-r_0}\right)}.
}
Note que la \'ultima ecuaci\'on tiene un notable parecido con la identidad de suma de la tangente.\\
Con lo anterior calculado, es directo calcular la siguiente cantidad:
\eq{
\tan\left(\theta+\frac{r_0\phi}{a-r_0}\right)&=\frac{\tan\theta+\tan\left(\frac{r_0\phi}{a-r_0}\right)}{1-\tan\theta\tan\left(\frac{r_0\phi}{a-r_0}\right)}\\
&=\frac{a\tan\left(\frac{a\phi}{a-r_0}\right)\left(1-\tan^2\left(\frac{r_0\phi}{a-r_0}\right)\right)}{r_0\left(1-\tan^2\left(\frac{r_0\phi}{a-r_0}\right)\right)}.
}
Finalmente:
\eq{
\tan\left(\theta+\frac{r_0\phi}{a-r_0}\right)=\frac{a\tan\left(\frac{a\phi}{a-r_0}\right)}{r_0}.
}
As\'i, $\theta(\phi)$ es:
\eq{
\theta(\phi)=\tan^{-1}\left(\frac{a}{r_0}\tan\left(\frac{a\phi}{a-r_0}\right)\right)-\frac{r_0\phi}{a-r_0}.
}

Ahora que conocemos $r(\phi)$ y $\theta(\phi)$ podemos calcular $r'(\theta)^2$ as\'i:
\eq{
\left(\frac{dr}{d\theta}\right)^2=\left(\frac{\frac{dr}{d\phi}}{\frac{d\theta}{d\phi}}\right)^2&=\left(\frac{-\frac{a(r_0^2-a^2)\sin\left(\frac{2a\phi}{a-r_0}\right)}{\sqrt{2}(a-r_0)\sqrt{a^2+r_0^2+(r_0^2-a^2)\cos\left(\frac{2a\phi}{a-r_0}\right)}}}{\frac{a^2\sec^2\left(\frac{a\phi}{a-r_0}\right)}{(a-r_0)r_0\left(\frac{a^2\tan^2\left(\frac{a\phi}{a-r_0}\right)}{r_0^2}+1\right)}-\frac{r_0}{a-r_0}}\right)^2\\
&=-\frac{a^2\left(-a^2-r_0^2+(a^2-r_0^2)\cos\left(\frac{2a\phi}{a-r_0}\right)\right)\tan^2\left(\frac{a\phi}{a-r_0}\right)}{2r_0^2}.
}
Por otro lado, si calculamos el lado izquierdo de la ecuaci\'on diferencial:
\eq{
\frac{a^2r^2}{a^2-r^2}\frac{r^2-r_0^2}{r_0^2}&=\frac{a^2\left(a^2+r_0^2+(r_0^2-a^2)\cos\left(\frac{2a\phi}{a-r_0}\right)\right)\left(\frac{1}{2}\left(a^2+r_0^2+(r_0^2-a^2)\cos\left(\frac{2a\phi}{a-r_0}\right)\right)-r_0^2\right)}{2r_0^2\left(a^2+\frac{1}{2}\left(-a^2-r_0^2-(r_0^2-a^2)\cos\left(\frac{2a\phi}{a-r_0}\right)\right)\right)}.
}
Que al simplificar se convierte en:
\eq{
-\frac{a^2\left(-a^2-r_0^2+(a-r_0)(a+r_0)\cos\left(\frac{2a\phi}{a-r_0}\right)\right)\tan^2\left(\frac{a\phi}{a-r_0}\right)}{2r_0^2}.
}
Lo que muestra que efectivamente la hipocicloide es la soluci\'on del problema.

Por \'ultimo queremos calcular el tiempo de viaje y la profundidad m\'axima alcanzada en funci\'on de la separaci\'on angular de los puntos sobre la Tierra.
Para encontrar las anteriores cantidades, primero notemos que $r_0\leq r\leq a$ y que $r_0=r \rightarrow \phi=0 \rightarrow \theta=0$, es decir, por la forma como construimos la soluci\'on, $r$ es m\'inimo en $\theta=0$. Ahora podemos buscar $\theta$ tal que $r=a$, para ello primero encontremos a $\phi$ que cumpla esa condici\'on:
\eq{
a^2=\frac{1}{2}\left(a^2+r_{0}^2+\left(r_{0}^2-a^2\right)\cos\left(\frac{2a\phi}{a-r_{0}}\right)\right)\rightarrow \phi=\frac{\pi}{2}\frac{a-r_0}{a}.
}
Si reemplazamos lo anterior en la ecuaci\'on para $\theta$ se obtiene que:
\[
\theta(r=a)=\frac{\pi}{2}\left(1-\frac{r_0}{a}\right).
\]
As\'i entonces, el anterior es el \'angulo que se barre entre ir de la superficie al punto de m\'aximo acercamiento; por razones de simetr\'ia, este debe ser igual al \'angulo que se barre en ir desde el punto de m\'aximo acercamiento al centro hasta la superficie de nuevo. De lo anterior se deduce entonces que si en el trayecto de viaje el veh\'iculo se acerca hasta el centro $r_0$, entonces los puntos est\'an separados angularmente una distancia:
\[
\alpha=\pi\left(1-\frac{r_0}{a}\right).
\]
Como es de esperarse, si los dos puntos son diametralmente opuestos entonces $r_0=0$ y la trayectoria pasa por el centro; adem\'as, como se comprueba f\'acilmente, es una l\'inea recta en la que se presenta movimiento arm\'onico simple.\\

Finalmente, la integral que da el tiempo de viaje se puede reparametrizar en t\'erminos de $\phi$ para obtener:
\eq{
t_{\min}&=\sqrt{\frac{m}{k}}\int_{-\frac{\pi}{2}\left(1-\frac{r_0}{a}\right)}^{\frac{\pi}{2}\left(1-\frac{r_0}{a}\right)}\sqrt{\frac{\left(\frac{dr}{d\phi}\right)^2+r^2(\phi)\left(\frac{d\theta}{d\phi}\right)^2}{a^2-r^2(\phi)}}\,d\phi,\\
&=\sqrt{\frac{m}{k}}\frac{\pi}{a}\sqrt{a^2-r_0^2}=\sqrt{\frac{m}{k}}\pi\sqrt{1-\left(\frac{r_0}{a}\right)^2}.
}
Pero $\frac{r_0}{a}=1-\frac{\alpha}{\pi}$, as\'i que el tiempo m\'inimo en t\'erminos de la separaci\'on angular es:
\[
t_{\min}=\sqrt{\frac{m}{k}}\pi\sqrt{1-\left(1-\frac{\alpha}{\pi}\right)^2}.
\]

Finalmente, se pide hallar el tiempo de viaje entre Los \'Angeles y Nueva York. Estas ciudades est\'an separadas por una distancia de $4800$ km. Adem\'as se sabe que el di\'ametro de la Tierra es de aproximadamente $a=6400$ km, luego su separaci\'on angular es:
\[
\alpha=\frac{4800}{6400}=0.75,
\]
adem\'as la masa de la Tierra es $M=6.0\cdot10^{24}$ kg y la constante de Cavendish tiene un valor de $G=6.7\cdot10^{-11}$ m$^3$ kg$^{-1}$ s$^{-2}$, entonces:
\[
\frac{k}{m}=1.5\cdot10^{-6}\,\text{s}^{-2}.
\]
El tiempo de viaje de Nueva York a Los \'Angeles es:
\[
t_{\text{NY-LA}}=1.64\cdot10^3\,\text{s}=27',
\]
y el radio m\'inimo es:
\[
r_0=a\left(1-\frac{0.75}{\pi}\right)=4.9\cdot10^3\,\text{km}.
\]

\subsection*{Ejercicio 2.16}
Para el presente problema el Lagrangiano est\'a dado por:
\eq{
L=e^{t\gamma}\left(\frac{1}{2}m\dot q(t)^2-\frac{1}{2}kq(t)^2\right).
\label{kl}
}
La ecuaci\'on de Euler-Lagrange para el anterior Lagrangiano es:
\eq{
e^{t\gamma}\left(kq(t)+m\left(\gamma\dot q(t)+\ddot q(t)\right)\right)=0,
}
que es la ecuaci\'on de movimiento de un oscilador arm\'onico amortiguado.
Si se hace la transformaci\'on de coordenadas $q=e^{-\frac{t\gamma}{2}}s$, el Lagrangiano toma la forma siguiente:
\eq{
L'=\frac{1}{8}\left(\left(m\gamma^2-4k\right)s(t)^2-4m\gamma\dot s(t)s(t)+4m\dot s(t)^2\right).
}
La ecuaci\'on de Euler-Lagrange es:
\eq{
\left(m\gamma^2-4k\right)s(t)=4m\ddot s(t),
}
esta es la ecuaci\'on de movimiento de un oscilador arm\'onico.
Finalmente, la funci\'on $h$ de Jacobi para el anterior Lagrangiano es constante de movimiento; si se expresa en t\'erminos de $q(t)$ queda (con $\omega^2=k/m$):
\eq{
h=\frac{1}{2}e^{t\gamma}\left(\omega^2q(t)^2+\gamma\dot q(t)q(t)+\dot q(t)^2\right),
}
que es la constante de movimiento para el oscilador arm\'onico amortiguado.

\subsection*{Ejercicio 2.18}
Para este caso, usando coordenadas esf\'ericas, tenemos lo siguiente:
\eq{
T&=\frac{1}{2}a^2m\left(\theta'^2+\sin^2(\theta)\phi'^2\right),\\
V&=mgz=mga\cos(\theta).
}
Pero $\phi=\omega t$. As\'i el Lagrangiano es:
\eq{
L=T-V=\frac{1}{2}a^2m\left(\theta'^2+\sin^2(\theta)\omega^2\right)-mga\cos(\theta).
}
La ecuaci\'on de Euler-Lagrange que satisface $\theta$ es:
\eq{
am\left(\left(a\cos(\theta)\omega^2+g\right)\sin(\theta)-a\theta''\right)=0.
\label{el}
}
Como se ve del Lagrangiano, la cantidad $h$ (la funci\'on de energ\'ia) es conservada ya que el Lagrangiano no depende expl\'icitamente del tiempo. Como adem\'as contiene $\theta$, el momento generalizado $p_\theta=a^2m\theta'$ no se conserva.
Para hallar condiciones de equilibrio hagamos $\theta'=0\Longrightarrow\theta''=0$. Esto implica en (\ref{el}) que:
\eq{
\left(a\cos(\theta)\omega^2+g\right)\sin(\theta)=0.
\label{iii}
}
La anterior ecuaci\'on implica o bien que $\theta=0$ o $\theta=\pi$ o que $\omega=\sqrt{\frac{-g}{a\cos(\theta)}}$. El m\'inimo valor que toma $\omega$ es $\omega_0=\sqrt{\frac{g}{a}}$. Para valores de $\omega$ menores que $\omega_0$, la ecuaci\'on (\ref{iii}) no tiene soluciones reales excepto las triviales (0 y $\pi$), es decir, no hay m\'as puntos de equilibrio.

\subsection*{Ejercicio 2.24}
\eq{
\label{ecu}
x&=\sum_{j=0}^\infty a_j\cos(j\omega t),\\
\dot x&=-\sum_{j=0}^\infty a_j\omega j\sin(j\omega t),\\
L&=\frac{m\dot x^2}{2}-\frac{kx^2}{2},\\
S&=\int_{0}^{\frac{2\pi}{\omega}}L\,dt.
}
Si sustituimos $x$ y $\dot x$ en $L$ obtenemos:
\eq{
L&=\frac{m\left(-\sum_{j=0}^\infty a_j\omega j\sin(j\omega t)\right)^2}{2}-\frac{k\left(\sum_{j=0}^\infty a_j\cos(j\omega t)\right)^2}{2},\\
L&=\frac{m\left(\sum_{j=0}^\infty a_j^2\omega^2j^2\sin^2(j\omega t)-2\sum_{i=0}^\infty\sum_{i<j}^\infty a_ja_i\omega^2ij\sin(j\omega t)\sin(i\omega t)\right)}{2}\\
&-\frac{k\left(\sum_{j=0}^\infty a_j^2\cos^2(j\omega t)+2\sum_{i=0}^\infty\sum_{i<j}^\infty a_ja_i\cos(j\omega t)\sin(i\omega t)\right)}{2}.\nonumber
}
Al hacer la integral, los t\'erminos cruzados con diferente \'indice se anulan, i.e., si $i\neq j$:
\eq{
\int_{0}^{\frac{2\pi}{\omega}}\sin(i\omega t)\sin(j\omega t)\,dt=\int_{0}^{\frac{2\pi}{\omega}}\cos(i\omega t)\cos(j\omega t)\,dt=0.
}
Por otro lado, las integrales con t\'erminos no cruzados son f\'acilmente calculadas as\'i:
\eq{
\int_{0}^{\frac{2\pi}{\omega}}\cos^2(0)\,dt&=\frac{2\pi}{\omega},\\
\int_{0}^{\frac{2\pi}{\omega}}\sin^2(0)\,dt&=0,\\
\int_{0}^{\frac{2\pi}{\omega}}\cos^2(j\omega t)\,dt&=\int_{0}^{\frac{2\pi}{\omega}}\sin^2(j\omega t)\,dt=\frac{\pi}{\omega}.
}
La acci\'on $S$ toma la siguiente forma:
\eq{
S&=\frac{m\sum_{j=1}^\infty a_j^2\omega^2j^2\frac{\pi}{\omega}}{2}-\frac{k\left(\frac{2\pi}{\omega}a_0^2+\sum_{j=1}^\infty a_j^2\frac{\pi}{\omega}\right)}{2},\\
S&=-\frac{k\pi}{\omega}a_0^2+\frac{\pi}{2}\sum_{j=1}^\infty a_j^2\left(m\omega j^2-\frac{k}{\omega}\right).
}
La acci\'on es ahora una funci\'on de las $a_j$, por lo tanto, para que sea extremo el gradiente $\nabla_a$ de la acci\'on con respecto a las $a_j$ debe ser $0$, es decir, el vector infinito con entradas:
\eq{
\left(\frac{-2k\pi a_0}{\omega},\dots,\pi a_j\left(mj^2\omega-\frac{k}{\omega}\right),\dots\right)
}
debe ser el vector nulo. Claramente lo anterior implica que $a_0=0$ y que:
\eq{
a_j\text{ o }\omega=\sqrt{\frac{k}{j^2m}}.
}
Supongamos que $a_l\neq 0$. Esto implica que $\omega=\sqrt{\frac{k}{l^2m}}$ y que $a_{j\neq l}=0$.
Reescribiendo la soluci\'on tenemos:
\eq{
x=a_l\cos\left(\frac{l}{l}\sqrt{\frac{k}{m}}t\right)=a_l\cos\left(\sqrt{\frac{k}{m}}t\right).
}

























