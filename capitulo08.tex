

\subsection*{Ejercicio 8.2}
Supongamos que tenemos un Lagrangiano $L(q_i,\dot q_i,t)$, con su respectivo Hamiltoniano $H$, y momentos conjugados $p_i$. Se desea averiguar como queda el nuevo Hamiltoniano $H'$ y los nuevos momentos conjugados $p_i'$ si se usa un nuevo Lagrangiano $L'=L+\frac{dF}{dt}$ donde $F$ es una funci\'on arbitraria pero diferenciable.
Los nuevos momentos generalizados vienen dados por:
\eq{
p'_i=\frac{\partial L'(q_i,\dot q_i,t)}{\partial \dot q_i}=\frac{\partial L}{\partial \dot q_i}+\frac{\partial \left(\frac{dF}{dt} \right)}{\partial \dot q_i}
}
Pero el \'ultimo t\'ermino puede transformarse asi: (Suma sobre \'indices repetidos):
\eq{
\frac{\partial \left(\frac{dF}{dt} \right)}{\partial \dot q_i}=\frac{\partial \left( \frac{\partial F}{\partial q_j} \dot q_j+\frac{\partial F}{\partial t} \right)}{\partial \dot q_i}=\frac{\partial F}{\partial q_i}
}
Teniendo en cuanto lo anterior y que $p_i=\frac{\partial L(q_i,\dot q_i,t)}{\partial \dot q_i}$ tenemos que:
\eq{
p'_i=p_i+\frac{\partial F}{\partial q_i}
}
El nuevo Hamiltoniano est\'a dado por: 
\eq{
H'&=\dot q_i p'_i-L'=\dot q_i \left(p_i+\frac{\partial F}{\partial q_i}\right)-L-\frac{dF}{dt}=\dot q_i \left(p_i+\frac{\partial F}{\partial q_i}\right)-L-\left(\frac{\partial F}{\partial q_i} \dot q_i+\frac{\partial F}{\partial t}\right)\\
H'&=\dot q_i p_i-L-\frac{\partial F}{\partial t}=H-\frac{\partial F}{\partial t}
}
Para mostrar que las ecuaciones que se obtienen con este nuevo Hamiltoniano para $p_i'$ y $q_i$ tienen la misma forma que las que se obtienen para $p_i$ y $q_i$ atrav\'es del Hamiltoniano original escribamos el principio de Hamilton para el antiguo Hamiltoniano $H$ teniendo en cuenta la adici\'on de $\frac{dF}{dt}$:\\
\eq{
\delta I&=\delta \int_{t_1}^{t_2} L+\frac{dF}{dt}dt=\delta \int_{t_1}^{t_2}\left(p_i \dot q_i-H+\frac{dF}{dt}\right)dt\\
&=\delta \int_{t_1}^{t_2}\left(p_i \dot q_i-H+\frac{\partial F}{\partial q_i} \dot q_i+\frac{\partial F}{\partial t}\right)dt  \nonumber
}
Los t\'erminos dentro de la integral se pueden reorganizar para obtener:
\eq{\delta \int_{t_1}^{t_2}\left(\left(p_i+\frac{\partial F}{\partial q_i} \right)\dot q_i-\left(H- \frac{\partial F}{\partial t}\right)\right)dt}
Pero el primer par\'entesis resulta ser $p_i'$ y el segundo $H'$ asi:
\eq{\delta \int_{t_1}^{t_2}\left(p_i'\dot q_i-H'\right)dt}
En este punto podemos asumir que $H'$ esta escrito en t\'erminos de las $q_i$ y las nuevas $p_i'$ y asi podemos usar las ecuaciones de Euler-Lagrange para hallar:
\eq{
\frac{d}{dt}\left( \frac{\partial f}{\partial \dot q_i}\right)-\frac{\partial f}{\partial q_i}&=\dot p_i'+\frac{\partial H'}{\partial q_i}=0\\
\frac{d}{dt}\left( \frac{\partial f}{\partial \dot p_i'}\right)-\frac{\partial f}{\partial p_i'}&=\dot q_i-\frac{\partial H'}{\partial p_i'}=0
}
Que son las ecuaciones de Hamilton para las variables $\left(q_i,p_i'\right)$\\
\\

Para el resto de ejercicios $a'$ denota la derivada total de la variable $a$ con respecto al tiempo $t$.\\


\subsection*{Ejercicio 8.9}
Para este problema se hace un procedimiento an\'alogo al que se hace en la formulaci\'on lagrangiana. Asi entonces
\eq{
\delta S=\delta \int_{t_1}^{t_2} q_i' p_i-H\left(q_j,p_j,t\right)-\lambda_i \psi_i\left(q_j,p_j,t\right)dt
}
La variaci\'on se puede hacer  ahora con $n$ $\delta q_i$, $n$ $\delta p_i$ y $m$ $\lambda_i$.
Para esta problema las $2n$ ecuaciones de Euler-Lagrange de interes son (siendo $f$ el integrando de la ecuaci\'on anterior y $k=1,...,n$):
\eq{
\frac{d}{dt}\left( \frac{\partial f}{\partial q_k'}\right)-\frac{\partial f}{\partial q_k}&=0\\
\frac{d}{dt}\left( \frac{\partial f}{\partial p_k'}\right)-\frac{\partial f}{\partial p_k}&=0
}
Sustituyendo para las primeras $n$ ecuaciones obtenemos:
\eq{
\frac{d}{dt}\left( \frac{\partial \left(q_i' p_i-H\left(q_j,p_j,t\right)-\lambda_i \psi_i\left(q_j,p_j,t\right)\right)}{\partial q_k'}\right) & \nonumber \\
-\frac{\partial \left(q_i' p_i-H\left(q_j,p_j,t\right)-\lambda_i \psi_i\left(q_j,p_j,t\right)\right)}{\partial q_k}&=0\\
\frac{d p_k}{dt}-\left(\frac{\partial \left(-H\left(q_j,p_j,t\right)-\lambda_i \psi_i\left(q_j,p_j,t\right)\right)}{\partial q_k}\right)&=0
}
Reorganizando t\'erminos  y teniendo en cuenta que hay suma sobre el \'indice repetido $i$ nos queda:
\eq{
-p'_k=\frac{\partial H}{\partial q_k}+\lambda_i \frac{\partial \psi_i\left(q_j,p_j,t\right)}{\partial q_k}
}
Para las restantes $n$ ecuaciones tenemos:
\eq{
\frac{d}{dt}\left( \frac{\partial \left(q_i' p_i-H\left(q_j,p_j,t\right)-\lambda_i \psi_i\left(q_j,p_j,t\right)\right)}{\partial p_k'}\right)\\
\quad \quad -\frac{\partial \left(q_i' p_i-H\left(q_j,p_j,t\right)-\lambda_i \psi_i\left(q_j,p_j,t\right)\right)}{\partial p_k}=0
\nonumber}
Reorganizando los t\'erminos tenemos (De nuevo suma sobre $i$):
\eq{0-\left(q'_k-\frac{\partial H}{\partial p_k}-\lambda_i \frac{\partial \psi_i\left(q_j,p_j,t\right)}{\partial p_k}\right)=0\\
q'_k =\frac{\partial H}{\partial p_k}+\lambda_i \frac{\partial \psi_i\left(q_j,p_j,t\right)}{\partial p_k} 
}

\subsection*{Ejercicio 8.12}
Usando la convenci\'on usada en el cap\'itulo 2 el lagrangiano del sistema se puede escribir como:
\eq{ L=\frac{m_1+m_2}{2} \textbf{r'}^2+\frac{1}{2}\frac{m_1 m_2}{m_1+m_2} \textbf{R'}^2-U\left(R\right)}
Usando coordenadas esfericas $R,\theta,\phi$ para $\textbf{R}$ y coordenadas rectangulares $x,y,z$ para $\textbf{r}$ y llamando $\mu$=$\frac{m_1 m_2}{m_1+m_2}$ y $M=m_1+m_2$ tenemos:
\eq{L=\frac{\mu}{2}\left(R'^2+R^2 \theta'^2+R^2 \sin^2 \left(\theta\right) \phi'^2\right)+\frac{M}{2}\left(x'^2+y'^2+z'^2\right)-U\left(R\right)}

De lo anterior se ve facilmente que el Hamiltoniano del sistema es:
\eq{H=\frac{1}{2\mu}\left(p_R^2+\frac{ p_\theta^2}{R^2}+\frac{p_\phi^2}{R^2 \sin^2 \left(\theta\right)}\right)+\frac{1}{2 M}\left(p_x^2+p_y^2+p_z^2\right)+U\left(R\right)}
Ahora obtegamos la ecuaciones de Hamilton para $p_x, p_y ,p_z, x, y ,z$ que son:
\eq{
p_x'&=-\frac{\partial H}{\partial x}=0\Rightarrow p_x=c_1\\
p_y'&=-\frac{\partial H}{\partial y}=0\Rightarrow p_y=c_2\\
p_z'&=-\frac{\partial H}{\partial z}=0\Rightarrow p_z=c_3\\
x'&=\frac{\partial H}{\partial p_x}=p_x/M\Rightarrow x=x_0+\left(p_x/M\right)t\\
y'&=\frac{\partial H}{\partial p_y}=p_x/M\Rightarrow y=y_0+\left(p_y/M\right)t\\
z'&=\frac{\partial H}{\partial p_z}=p_x/M\Rightarrow z=z_0+\left(p_z/M\right)t
}
Con lo anterior nos hemos librado de la mitad de las variables y podemos escribir el Hamiltoniano de la siguiente manera:
\eq{
 H=\frac{1}{2\mu}\left(p_R^2+\frac{ p_\theta^2}{R^2}+\frac{p_\phi^2}{R^2 \sin^2 \left(\theta\right)}\right)+U\left(R\right)+c
}
Donde $c=\frac{1}{2 M}\left(p_x^2+p_y^2+p_z^2\right)$. De ahora en adelante dicha constante se omitira ya que no afecta las ecuaciones de movimiento de las dem\'as variables. En el anterior Hamiltoniano la variable $\phi$ es c\'iclica por lo que $p_\phi=l_z$ es constante de movimiento y asi:
\eq{
H&=\frac{1}{2\mu}\left(p_R^2+\frac{ p_\theta^2}{R^2}+\frac{l_z^2}{R^2 \sin^2 \left(\theta\right)}\right)+U\left(R\right)\\
&= \frac{1}{2\mu}\left(p_R^2+\frac{1}{R^2}\left(p_\theta^2+\frac{l_z^2}{\sin^2 \left(\theta\right)}\right)\right)+U\left(R\right)\\
&=\frac{1}{2\mu} \left(p_R^2+\frac{f\left(\theta,p_\theta\right)}{R^2}\right)+U\left(R\right)
}
Ahora como en ves de $\theta$ y $p_\theta$ por separado aparece una funci\'on  $f\left(\theta,p_\theta\right)$ de las dos esta debe ser una constante de movimientom \emph{i.e.} 
\eq{
f\left(\theta,p_\theta\right)=L^2=p_\theta^2+\frac{l_z^2}{\sin^2 \left(\theta\right)}=k
}
De la anterior ecuaci\'on podemos obtener $p_\theta$ en t\'erminos de $\theta$ asi:
\eq{
p_\theta=\pm \sqrt{L^2-\frac{l_z^2}{\sin^2\left(\theta\right)}}
} 
y ademas podemos escribir el Hamiltoniano asi:
\eq{
H=\frac{1}{2\mu}\left(p_r^2+\frac{L^2}{R^2}\right)+U\left(r\right)
}
Finalmente podemos obtener $p_R$ en funci\'on de $r$ como ($H=E=$ constante de Movimiento):
\eq{
p_R=\pm \sqrt{2 \mu \left(H-U\left(r\right)-\frac{L^2}{2 R^2}\right)}=\pm \sqrt{2 \mu \left(E-U\left(r\right)-\frac{L^2}{2 R^2}\right)}  
}

Con lo anterior en mente podemos escribir las restantes ecuaciones de Hamilton:
\eq{
p_R'&=-\frac{\partial H}{\partial R}=\frac{L^2}{\mu R^3}-\frac{\partial U}{\partial R}\Rightarrow p_R-p_{R_0}=\int_{0}^{t} \frac{L^2}{\mu R^3}-\frac{\partial U}{\partial R} dt\\
p_\theta'&=-\frac{\partial H}{\partial \theta}=\frac{\cos\left(\theta\right)}{\sin^3\left(\theta\right)}\frac{l_z^2}{\mu R^2}\Rightarrow p_\theta-p_{\theta_0}=\int_{0}^{t} \frac{\cos\left(\theta\right)}{\sin^3\left(\theta\right)}\frac{l_z^2}{\mu R^2} dt\\
p_\phi'&=-\frac{\partial H}{\partial \phi}=0\Rightarrow p_\phi=l_z\\
R'&=\frac{\partial H}{\partial p_R}=\frac{p_r}{\mu} \Rightarrow t=\int_{R_0}^{R} \frac{dR}{\sqrt{\frac{2 }{\mu}\left(E-U-\frac{L^2}{2mR^2}\right)}}\\
\theta'&=\frac{\partial H}{\partial p_\theta}=\frac{p_\theta}{\left(R^2 \mu\right)}\Rightarrow \int_0^t \frac{dt}{\mu R^2}=\int_{\theta_0}^{\theta}\frac{d\theta}{\sqrt{L^2-\frac{l_z^2}{\sin^2{\theta}}}}\\
\phi'&=\frac{\partial H}{\partial p_\phi}=\frac{l_z}{\left(\mu R^2 \sin^2\left(\theta\right)\right)}\Rightarrow \phi-\phi_0=l_z \int_0^t \frac{dt}{m R^2 \sin^2\left(\theta\right)}
}
Note que una vez realizada la integral correspodiente a $R'$ y obtenido $R\left(t\right)$ en forma explicita este se puede reemplazar en la integral correspodiente a $\theta'$ y asi obtener $\theta$ como funci\'on explicita de $t$. Con estas dos variables conocidas se puede sustituir en la ecuaciones de las dem\'as variables, realizar las respectivas integraciones e inversiones y resolver el problema completamente.




\subsection*{Ejercicio 8.14}
Sea $L=a x'^2+\frac{b y'}{x}+c x' y'+ f y^2x' z' +g y'-k \sqrt{x^2+y^2}$. De este obtenemos los momentos conjugados:
\eq{
p_x&=f z' y^2+2 a x'+c y'\\
p_y&=\frac{b}{x}+g+c x' \\
p_z&=f y^2 x'
}
Con estos podemos obtener el Hamiltoniano:
\eq{
H=&q_i' p_i-L=\left(f x' z' y^2+\left(\frac{b}{x}+g+c x'\right) y'+x' \left(f z' y^2+2 a x'+c y'\right)\right)\\
&-\left(a x'^2+\frac{b y'}{x}+c x' y'+ f y^2x' z' +g y'-k \sqrt{x^2+y^2}\right)\nonumber\\
H=&\left(2 f x' z' y^2+2 a x'^2+g y'+2 c x' y'+\frac{b y'}{x}\right)\\
&-\left(a x'^2+\frac{b y'}{x}+c x' y'+ f y^2x' z' +g y'-k \sqrt{x^2+y^2}\right)\nonumber\\
H=&a x'^2+\left(f z' y^2+c y'\right) x'+k \sqrt{x^2+y^2}
}
Aunque la transformaci\'on (lineal) que manda a las velocidades en los momentos no tiene inversa lo que si se puede hacer es de la ecuaci\'on de $p_x$ despejar $y'$ y de la de $p_z$ a $x'$ para obtener:
\eq{
y'&=\frac{-f z' y^2+p_x-2 a x'}{c}=\frac{-f z' y^2+p_x-2 a \frac{p_z}{f y^2}}{c}\\
x'&=\frac{p_z}{f y^2}
}

y sustituirlos en la ecuaci\'on del Hamiltoniano:
\eq{
H&=a x'^2+\left(f z' y^2+c \frac{-f z' y^2+p_x-2 a x'}{c}\right) x'+k \sqrt{x^2+y^2}\\
H&=\frac{a p_z^2}{f^2 y^4}+\frac{\left(p_x-\frac{2 a p_z}{f y^2}\right) p_z}{f y^2}+k \sqrt{x^2+y^2}\\
H&=-\frac{a p_z^2}{f^2 y^4}+\frac{p_x p_z}{f y^2}+k \sqrt{x^2+y^2}
}
Este Hamiltoniano no depende ni de $p_y$ ni de $z$, por lo tanto $y=c$ y $p_z=k$ son constantes de movimiento. Adem\'as $H$ no depende explicitamente de $t$ por lo tanto $H$ tambien es constante de movimiento


\subsection*{Ejercicio 8.19}
Del dibujo se ve que la posici\'on del cuerpo puede ser escrita asi:
\eq{
\tilde{x}&=l \sin \left(\theta \right)+x\\
\tilde{z}&=-l \cos \left(\theta \right)+z=a x^2-l \cos \left(\theta \right)
}
Donde $\theta$ es el \'angulo que forma el eje del p\'endulo con la vertical. As\'i el lagrangiano toma la siguiente forma:
\eq{
L=\frac{1}{2} m \left(\tilde{x}'^2+\tilde{z}'^2\right)-m g \tilde{z}
}
Usando como coordenadas generalizadas $x$ y $\theta $ se reescribe asi:
\eq{
\frac{1}{2} m \left(\left(x'+l \cos \left(\theta \right) \theta '\right)^2+\left(2 a x x'+l
   \sin \left(\theta \right) \theta '\right)^2\right)- m g \left(a x^2-l \cos \left(\theta \right)\right)
}
Expandiendo:
\eq{
L=&2 a^2 m x'^2 x^2-a g m x^2+2 a l m \sin \left(\theta \right) x' \theta ' x+\frac{1}{2}
   m x'^2\\ 
&+\frac{1}{2} l^2 m \cos ^2\left(\theta \right) \theta '^2+\frac{1}{2} l^2 m \sin ^2\left(\theta
   \right) \theta '^2+g l m \cos \left(\theta \right)+l m \cos \left(\theta \right) x' \theta '\nonumber
}
Lo que puede ser reescrito asi:
\eq{
L=L_0+\frac{1}{2} \textbf{q'}^T T \textbf{q'}
}
con:
\eq{
\textbf{q'}&=\left(
\begin{array}{l}%[pos]{spalten}
x'\\
\theta'
\end{array} \right)\\
T&=
\left(
\begin{array}{ll}
 4 a^2 m x^2+m & l m \cos \left(\theta \right)+2 a l m \sin \left(\theta \right) x \\
 l m \cos \left(\theta \right)+2 a l m \sin \left(\theta \right) x & l^2 m
\end{array}
\right)\\
L_0&=g l m \cos \left(\theta \right)-a g m x^2 
}

Para hallar el Hamiltoniano se usa el procedimiento  de la ecuaci\'on 8.27 p\'agina 340 cap\'itulo 8:
\eq{
H\left(q,p,t\right)=\frac{1}{2} \textbf{p}^T T^{-1} \textbf{p}-L_0\left(q,t\right)
}
Para este caso la inversa de $T$ es:
\eq{
T^{-1}=&\frac{1}{\det\left(T\right)} \left(
\begin{array}{ll}
 l^2 m & -2 a l m \sin \left(\theta \right) x-l m \cos \left(\theta \right) \\
 -2 a l m \sin \left(\theta \right) x-l m \cos \left(\theta \right) & 4 a^2 m x^2+m
\end{array}
\right)\\
\det\left(T\right)=&l^2 m^2-l^2 \cos ^2\left(\theta \right) m^2+4 a^2 l^2 x^2 m^2-4 a^2 l^2 \sin ^2\left(\theta \right) x^2
   m^2\\
&-4 a l^2 \cos \left(\theta \right) \sin \left(\theta \right) x m^2 \nonumber\\
\det\left(T\right)=&m^2 l^2 \left(\sin\left(\theta\right)^2+4 a^2 x^2 \cos^2\left(\theta\right)-4 a x \cos \left(\theta \right) \sin \left(\theta \right) \right)\\
\det\left(T\right)=&m^2 l^2 \left(\sin\left(\theta\right)-2 a x \cos\left(\theta\right)\right)^2
}

\eq{
\frac{\textbf{p}^T T^{-1} \textbf{p}}{2}=\frac{m \left(l^2 p_x^2-2 l p_{\theta } p_x \left(\cos \left(\theta \right)+2 a \sin \left(\theta \right) x\right)+p_{\theta }^2 \left(4 a^2 x^2+1\right)\right)}{2 \det \left(T\right)}
}
Finalmente el Hamiltoniano queda asi:
\eq{
H&=\frac{m \left(l^2 p_x^2-2 l p_{\theta } p_x \left(\cos \left(\theta \right)+2 a \sin \left(\theta \right) x\right)+p_{\theta }^2 \left(4 a^2 x^2+1\right)\right)}{2 \det \left(T\right)}-m g l \cos \left(\theta \right)+m a g x^2\\
H&=\frac{l^2 p_x^2-2 l p_{\theta } p_x \left(\cos \left(\theta \right)+2 a \sin \left(\theta \right) x\right) +p_{\theta }^2 \left(4 a^2 x^2+1\right)}{2 l^2 m \left(\sin \left(\theta \right)-2 a \cos \left(\theta \right) x\right)^2}-m g l\cos \left(\theta \right)+ m g a x^2
}
Ahora con el hamiltoniano hallamos las ecuaciones de Hamilton:

\eq{x'=\frac{\partial H}{\partial p_x}=&\frac{l p_x-\left(\cos \left(\theta \right)+2 a \sin \left(\theta \right) x\right) p_{\theta }}{l m \left(\sin \left(\theta\right)-2 a \cos \left(\theta \right) x\right)^2}\\
\theta'=\frac{\partial H}{\partial p_\theta}=&\frac{\left(4 a^2 x^2+1\right) p_{\theta }-l \left(\cos \left(\theta \right)+2 a \sin \left(\theta \right) x\right) p_x}{l^2 m \left(\sin \left(\theta \right)-2 a \cos \left(\theta \right) x\right)^2}\\
p_x'=-\frac{\partial H}{\partial x}=&\frac{2 a}{m} \bigg( -g x m^2+\frac{p_{\theta } \left(l \sin \left(\theta \right) p_x-2 a x p_{\theta   }\right)}{l^2 \left(\sin \theta -2 a x \cos \theta \right)^2}  \\
&-\frac{\cos \left(\theta \right) \left(l^2 p_x^2-2 l \left(\cos \left(\theta \right)+2 a \sin \left(\theta \right) x\right) p_{\theta }  p_x+\left(4 a^2 x^2+1\right) p_{\theta }^2\right)} {l^2 \left(\sin \left(\theta \right)-2 a \cos \left(\theta \right) x\right)^3} \bigg) 
\nonumber}

\eq{
p_\theta'=-\frac{\partial H}{\partial \theta}=& \frac{1}{l^2 m} \left(-g m^2 \sin \left(\theta \right) l^3+\frac{p_x p_{\theta } l}{2 a \cos \left(\theta \right) x-\sin\left(\theta \right)} \right)\\
 &+\frac{1}{l^2 m} \frac{\left(\cos \left(\theta \right)+2 a \sin \left(\theta \right) x\right) }{\left(\sin \left(\theta \right)-2 a \cos \left(\theta \right) x\right)^3} \nonumber\\
&\quad \quad \times \left(l^2 p_x^2-2 l\left(\cos \left(\theta \right)+2 a \sin \left(\theta \right) x\right) p_{\theta } p_x+\left(4 a^2 x^2+1\right) p_{\theta }^2\right)\nonumber
}

La anteriores expresi\'ones para $p_x'$ y $p_\theta'$ se pueden reorganizar asi:
\eq{
p_\theta'=&\frac{\left(\cos \left(\theta \right)+2 a \sin \left(\theta \right) x\right) p_x^2}{m \left(\sin \left(\theta \right)-2 a \cos\left(\theta \right) x\right)^3}\\ 
&+\left(\frac{2 a \cos \left(\theta \right) x-\sin \left(\theta \right)}{l m \left(\sin\left(\theta \right)-2 a \cos \left(\theta \right) x\right)^2}-\frac{2 \left(\cos \left(\theta \right)+2 a \sin \left(\theta \right)x\right)^2}{l m \left(\sin \left(\theta \right)-2 a \cos \left(\theta \right) x\right)^3}\right) p_{\theta }p_x\nonumber\\ 
&+\frac{\left(\cos \left(\theta \right)+2 a \sin \left(\theta \right) x\right) \left(4 a^2 x^2+1\right)p_{\theta }^2}{l^2 m \left(\sin \left(\theta \right)-2 a \cos \left(\theta \right) x\right)^3}-g l m \sin \left(\theta\right) \nonumber
}
\eq{
p_x'=&-\frac{2 a \cos \left(\theta \right) p_x^2}{m \left(\sin \left(\theta \right)-2 a \cos \left(\theta \right)
   x\right)^3}\\ 
&+\left(\frac{2 a \sin \left(\theta \right)}{l m \left(\sin \left(\theta \right)-2 a \cos \left(\theta \right)
   x\right)^2}+\frac{4 a \cos \left(\theta \right) \left(\cos \left(\theta \right)+2 a \sin \left(\theta \right) x\right)}{l m \left(\sin
   \left(\theta \right)-2 a \cos \left(\theta \right) x\right)^3}\right) p_{\theta } p_x \nonumber \\ 
&+\left(-\frac{4 x
   a^2}{l^2 m \left(\sin \left(\theta \right)-2 a \cos \left(\theta \right) x\right)^2}-\frac{2 \cos \left(\theta \right) \left(4
   a^2 x^2+1\right) a}{l^2 m \left(\sin \left(\theta \right)-2 a \cos \left(\theta \right) x\right)^3}\right)
   p_{\theta }^2-2 a g m x \nonumber
}
