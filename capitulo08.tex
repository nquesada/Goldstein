\documentclass[letterpaper,10pt]{article}
\usepackage[latin1]{inputenc}
\usepackage[dvips]{graphicx}
\usepackage[spanish]{babel}


\textwidth = 16.5 cm
\textheight = 23.5 cm
\oddsidemargin = 0.0 cm
\evensidemargin = 0.0 cm
\topmargin = 0.0 cm
\headheight = 0.0 cm
\headsep = 0.0 cm 

%opening
\title{Ejercicios del cap\'itulo 8 de\\ \emph{Classical Mechanics} de H. Goldstein}
\author{Nicol\'as Quesada M. \\ {\small \sf Instituto de F\'isica, Universidad de Antioquia}}
\date{}

\begin{document}

\maketitle

\section*{Ejercicio 8.2}
Supongamos que tenemos un Lagrangiano $L(q_i,\dot q_i,t)$, con su respectivo Hamiltoniano $H$, y momentos conjugados $p_i$. Se desea averiguar como queda el nuevo Hamiltoniano $H'$ y los nuevos momentos conjugados $p_i'$ si se usa un nuevo Lagrangiano $L'=L+\frac{dF}{dt}$ donde $F$ es una funci\'on arbitraria pero diferenciable.
Los nuevos momentos generalizados vienen dados por:
$$
p'_i=\frac{\partial L'(q_i,\dot q_i,t)}{\partial \dot q_i}=\frac{\partial L}{\partial \dot q_i}+\frac{\partial \left(\frac{dF}{dt} \right)}{\partial \dot q_i}
$$
Pero el \'ultimo t\'ermino puede transformarse asi: (Suma sobre \'indices repetidos):
$$
\frac{\partial \left(\frac{dF}{dt} \right)}{\partial \dot q_i}=\frac{\partial \left( \frac{\partial F}{\partial q_j} \dot q_j+\frac{\partial F}{\partial t} \right)}{\partial \dot q_i}=\frac{\partial F}{\partial q_i}
$$
Teniendo en cuanto lo anterior y que $p_i=\frac{\partial L(q_i,\dot q_i,t)}{\partial \dot q_i}$ tenemos que:
$$
p'_i=p_i+\frac{\partial F}{\partial q_i}
$$
El nuevo Hamiltoniano est\'a dado por: 
$$
H'=\dot q_i p'_i-L'=\dot q_i (p_i+\frac{\partial F}{\partial q_i})-L-\frac{dF}{dt}=\dot q_i (p_i+\frac{\partial F}{\partial q_i})-L-(\frac{\partial F}{\partial q_i} \dot q_i+\frac{\partial F}{\partial t})$$ $$
H'=\dot q_i p_i-L-\frac{\partial F}{\partial t}=H-\frac{\partial F}{\partial t}
$$
Para mostrar que las ecuaciones que se obtienen con este nuevo Hamiltoniano para $p_i'$ y $q_i$ tienen la misma forma que las que se obtienen para $p_i$ y $q_i$ atrav\'es del Hamiltoniano original escribamos el principio de Hamilton para el antiguo Hamiltoniano $H$ teniendo en cuenta la adici\'on de $\frac{dF}{dt}$:\\
$$\delta I=\delta \int_{t_1}^{t_2} L+\frac{dF}{dt}dt=\delta \int_{t_1}^{t_2}(p_i \dot q_i-H+\frac{dF}{dt})dt=\delta \int_{t_1}^{t_2}(p_i \dot q_i-H+\frac{\partial F}{\partial q_i} \dot q_i+\frac{\partial F}{\partial t})dt$$
Los t\'erminos dentro de la integral se pueden reorganizar para obtener:
$$ \delta \int_{t_1}^{t_2}((p_i+\frac{\partial F}{\partial q_i} )\dot q_i-(H- \frac{\partial F}{\partial t}))dt$$
Pero el primer par\'entesis resulta ser $p_i'$ y el segundo $H'$ asi:
$$ \delta \int_{t_1}^{t_2}(p_i'\dot q_i-H')dt$$
En este punto podemos asumir que $H'$ esta escrito en t\'erminos de las $q_i$ y las nuevas $p_i'$ y asi podemos usar las ecuaciones de Euler-Lagrange para hallar:
$$
\frac{d}{dt}\left( \frac{\partial f}{\partial \dot q_i}\right)-\frac{\partial f}{\partial q_i}=\dot p_i'+\frac{\partial H'}{\partial q_i}=0$$ 
$$
\frac{d}{dt}\left( \frac{\partial f}{\partial \dot p_i'}\right)-\frac{\partial f}{\partial p_i'}=\dot q_i-\frac{\partial H'}{\partial p_i'}=0
$$
Que son las ecuaciones de Hamilton para las variables $(q_i,p_i')$\\
\\

Para el resto de ejercicios $a'$ denota la derivada total de la variable $a$ con respecto al tiempo $t$.\\


\section*{Ejercicio 8.9}
Para este problema se hace un procedimiento an\'alogo al que se hace en la formulaci\'on lagrangiana. Asi entonces:
$$
\delta S=\delta \int_{t_1}^{t_2} q_i' p_i-H(q_j,p_j,t)-\lambda_i \psi_i(q_j,p_j,t)dt
$$
La variaci\'on se puede hacer  ahora con $n$ $\delta q_i$, $n$ $\delta p_i$ y $m$ $\lambda_i$.
Para esta problema las $2n$ ecuaciones de Euler-Lagrange de interes son (siendo $f$ el integrando de la ecuaci\'on anterior y $k=1,...,n$):
$$
\frac{d}{dt}\left( \frac{\partial f}{\partial q_k'}\right)-\frac{\partial f}{\partial q_k}=0$$ $$
\frac{d}{dt}\left( \frac{\partial f}{\partial p_k'}\right)-\frac{\partial f}{\partial p_k}=0
$$
Sustituyendo para las primeras $n$ ecuaciones obtenemos:
$$
\frac{d}{dt}\left( \frac{\partial (q_i' p_i-H(q_j,p_j,t)-\lambda_i \psi_i(q_j,p_j,t))}{\partial q_k'}\right)-\frac{\partial (q_i' p_i-H(q_j,p_j,t)-\lambda_i \psi_i(q_j,p_j,t))}{\partial q_k}=0$$ $$
\frac{d p_k}{dt}-(\frac{\partial (-H(q_j,p_j,t)-\lambda_i \psi_i(q_j,p_j,t))}{\partial q_k})=0
$$
Reorganizando t\'erminos  y teniendo en cuenta que hay suma sobre el \'indice repetido $i$ nos queda:
$$
-p'_k=\frac{\partial H}{\partial q_k}+\lambda_i \frac{\partial \psi_i(q_j,p_j,t)}{\partial q_k}
$$
Para las restantes $n$ ecuaciones tenemos:
$$
\frac{d}{dt}\left( \frac{\partial (q_i' p_i-H(q_j,p_j,t)-\lambda_i \psi_i(q_j,p_j,t))}{\partial p_k'}\right)-\frac{\partial (q_i' p_i-H(q_j,p_j,t)-\lambda_i \psi_i(q_j,p_j,t))}{\partial p_k}=0
$$
Reorganizando los t\'erminos tenemos (De nuevo suma sobre $i$):
$$0-\left(q'_k-\frac{\partial H}{\partial p_k}-\lambda_i \frac{\partial \psi_i(q_j,p_j,t)}{\partial p_k}\right)=0$$ o $$q'_k =\frac{\partial H}{\partial p_k}+\lambda_i \frac{\partial \psi_i(q_j,p_j,t)}{\partial p_k} $$

\section*{Ejercicio 8.12}
Usando la convenci\'on usada en el cap\'itulo 2 el lagrangiano del sistema se puede escribir como:
$$ L=\frac{m_1+m_2}{2} \textbf{r'}^2+\frac{1}{2}\frac{m_1 m_2}{m_1+m_2} \textbf{R'}^2-U(R)$$
Usando coordenadas esfericas $R,\theta,\phi$ para $\textbf{R}$ y coordenadas rectangulares $x,y,z$ para $\textbf{r}$ y llamando $\mu$=$\frac{m_1 m_2}{m_1+m_2}$ y $M=m_1+m_2$ tenemos:

$$ L=\frac{\mu}{2}(R'^2+R^2 \theta'^2+R^2 \sin^2 (\theta) \phi'^2)+\frac{M}{2}(x'^2+y'^2+z'^2)-U(R)$$

De lo anterior se ve facilmente que el Hamiltoniano del sistema es:
$$H=\frac{1}{2\mu}(p_R^2+\frac{ p_\theta^2}{R^2}+\frac{p_\phi^2}{R^2 \sin^2 (\theta)})+\frac{1}{2 M}(p_x^2+p_y^2+p_z^2)+U(R)$$
Ahora obtegamos la ecuaciones de Hamilton para $p_x, p_y ,p_z, x, y ,z$ que son:
$$p_x'=-\frac{\partial H}{\partial x}=0\Rightarrow p_x=c_1$$
$$p_y'=-\frac{\partial H}{\partial y}=0\Rightarrow p_y=c_2$$
$$p_z'=-\frac{\partial H}{\partial z}=0\Rightarrow p_z=c_3$$
$$x'=\frac{\partial H}{\partial p_x}=p_x/M\Rightarrow x=x_0+(p_x/M)t$$
$$y'=\frac{\partial H}{\partial p_y}=p_x/M\Rightarrow y=y_0+(p_y/M)t$$
$$z'=\frac{\partial H}{\partial p_z}=p_x/M\Rightarrow z=z_0+(p_z/M)t$$
Con lo anterior nos hemos librado de la mitad de las variables y podemos escribir el Hamiltoniano de la siguiente manera:
$$ H=\frac{1}{2\mu}(p_R^2+\frac{ p_\theta^2}{R^2}+\frac{p_\phi^2}{R^2 \sin^2 (\theta)})+U(R)+c$$
Donde $c=\frac{1}{2 M}(p_x^2+p_y^2+p_z^2)$. De ahora en adelante dicha constante se omitira ya que no afecta las ecuaciones de movimiento de las dem\'as variables. En el anterior Hamiltoniano la variable $\phi$ es c\'iclica por lo que $p_\phi=l_z$ es constante de movimiento y asi:

$$ H=\frac{1}{2\mu}(p_R^2+\frac{ p_\theta^2}{R^2}+\frac{l_z^2}{R^2 \sin^2 (\theta)})+U(R)= \frac{1}{2\mu}(p_R^2+\frac{1}{R^2}(p_\theta^2+\frac{l_z^2}{\sin^2 (\theta)}))+U(R)=\frac{1}{2\mu}(p_R^2+\frac{f(\theta,p_\theta)}{R^2}))+U(R)$$
Ahora como en ves de $\theta$ y $p_\theta$ por separado aparece una funci\'on  $f(\theta,p_\theta)$ de las dos esta debe ser una constante de movimientom \emph{i.e.} 
$$f(\theta,p_\theta)=L^2=p_\theta^2+\frac{l_z^2}{\sin^2 (\theta)}=k$$
De la anterior ecuaci\'on podemos obtener $p_\theta$ en t\'erminos de $\theta$ asi:
$$p_\theta=\pm \sqrt{L^2-\frac{l_z^2}{\sin^2(\theta)}}$$ y ademas podemos escribir el Hamiltoniano asi:
$$H=\frac{1}{2\mu}(p_r^2+\frac{L^2}{R^2})+U(r) $$
Finalmente podemos obtener $p_R$ en funci\'on de $r$ como ($H=E=$ constante de Movimiento):
$$p_R=\pm \sqrt{2 \mu (H-U(r)-\frac{L^2}{2 R^2})}=\pm \sqrt{2 \mu (E-U(r)-\frac{L^2}{2 R^2})}  $$

Con lo anterior en mente podemos escribir las restantes ecuaciones de Hamilton:
$$p_R'=-\frac{\partial H}{\partial R}=\frac{L^2}{\mu R^3}-\frac{\partial U}{\partial R}\Rightarrow p_R-p_{R_0}=\int_{0}^{t} \frac{L^2}{\mu R^3}-\frac{\partial U}{\partial R} dt$$
$$p_\theta'=-\frac{\partial H}{\partial \theta}=\frac{\cos(\theta)}{\sin^3(\theta)}\frac{l_z^2}{\mu R^2}\Rightarrow p_\theta-p_{\theta_0}=\int_{0}^{t} \frac{\cos(\theta)}{\sin^3(\theta)}\frac{l_z^2}{\mu R^2} dt$$
$$p_\phi'=-\frac{\partial H}{\partial \phi}=0\Rightarrow p_\phi=l_z$$
$$R'=\frac{\partial H}{\partial p_R}=p_r/\mu \Rightarrow t=\int_{R_0}^{R} \frac{dR}{\sqrt{\frac{2 }{\mu}(E-U-\frac{L^2}{2mR^2})}}$$
$$\theta'=\frac{\partial H}{\partial p_\theta}=p_\theta/(R^2 \mu)\Rightarrow \int_0^t \frac{dt}{\mu R^2}=\int_{\theta_0}^{\theta}\frac{d\theta}{\sqrt{L^2-\frac{l_z^2}{\sin^2{\theta}}}}$$
$$\phi'=\frac{\partial H}{\partial p_\phi}=l_z/(\mu R^2 \sin^2(\theta))\Rightarrow \phi-\phi_0=l_z \int_0^t \frac{dt}{m R^2 \sin^2(\theta)}$$
Note que una vez realizada la integral correspodiente a $R'$ y obtenido $R(t)$ en forma explicita este se puede reemplazar en la integral correspodiente a $\theta'$ y asi obtener $\theta$ como funci\'on explicita de $t$. Con estas dos variables conocidas se puede sustituir en la ecuaciones de las dem\'as variables, realizar las respectivas integraciones e inversiones y resolver el problema completamente.




\section*{Ejercicio 8.14}
Sea $L=a x'^2+\frac{b y'}{x}+c x' y'+ f y^2x' z' +g y'-k \sqrt{x^2+y^2}$. De este obtenemos los momentos conjugados:
$$p_x=f z' y^2+2 a x'+c y'$$
$$p_y=\frac{b}{x}+g+c x'$$ 
$$p_z=f y^2 x'$$
Con estos podemos obtener el Hamiltoniano:
$$
H=q_i' p_i-L=(f x' z' y^2+\left(\frac{b}{x}+g+c x'\right) y'+x' \left(f z' y^2+2 a x'+c y'\right))$$ $$-(a x'^2+\frac{b y'}{x}+c x' y'+ f y^2x' z' +g y'-k \sqrt{x^2+y^2})$$ $$
H=(2 f x' z' y^2+2 a x'^2+g y'+2 c x' y'+\frac{b y'}{x})-(a x'^2+\frac{b y'}{x}+c x' y'+ f y^2x' z' +g y'-k \sqrt{x^2+y^2})$$ $$
H=a x'^2+\left(f z' y^2+c y'\right) x'+k \sqrt{x^2+y^2}
$$
Aunque la transformaci\'on (lineal) que manda a las velocidades en los momentos no tiene inversa lo que si se puede hacer es de la ecuaci\'on de $p_x$ despejar $y'$ y de la de $p_z$ a $x'$ para obtener:
$$
y'=\frac{-f z' y^2+p_x-2 a x'}{c}=\frac{-f z' y^2+p_x-2 a \frac{p_z}{f y^2}}{c}$$ $$
x'=\frac{p_z}{f y^2}
$$

y sustituirlos en la ecuaci\'on del Hamiltoniano:
$$
H=a x'^2+\left(f z' y^2+c \frac{-f z' y^2+p_x-2 a x'}{c}\right) x'+k \sqrt{x^2+y^2}$$ $$
H=\frac{a p_z^2}{f^2 y^4}+\frac{\left(p_x-\frac{2 a p_z}{f y^2}\right) p_z}{f y^2}+k \sqrt{x^2+y^2}$$ $$
H=-\frac{a p_z^2}{f^2 y^4}+\frac{p_x p_z}{f y^2}+k \sqrt{x^2+y^2}
$$
Este Hamiltoniano no depende ni de $p_y$ ni de $z$, por lo tanto $y=c$ y $p_z=k$ son constantes de movimiento. Adem\'as $H$ no depende explicitamente de $t$ por lo tanto $H$ tambien es constante de movimiento


\section*{Ejercicio 8.19}
Del dibujo se ve que la posici\'on del cuerpo puede ser escrita asi:
$$
\tilde{x}=l \sin (\theta )+x$$ $$
\tilde{z}=-l \cos (\theta )+z=a x^2-l \cos (\theta )
$$
Donde $\theta$ es el \'angulo que forma el eje del p\'endulo con la vertical. As\'i el lagrangiano toma la siguiente forma:
$$
L=\frac{1}{2} m \left(\tilde{x}'^2+\tilde{z}'^2\right)-m g \tilde{z}
$$
Usando como coordenadas generalizadas $x$ y $\theta $ se reescribe asi:
$$
\frac{1}{2} m \left(\left(x'+l \cos (\theta ) \theta '\right)^2+\left(2 a x x'+l
   \sin (\theta ) \theta '\right)^2\right)- m g \left(a x^2-l \cos (\theta )\right)
$$
Expandiendo:
$$
L=2 a^2 m x'^2 x^2-a g m x^2+2 a l m \sin (\theta ) x' \theta ' x+\frac{1}{2}
   m x'^2+\\ \frac{1}{2} l^2 m \cos ^2(\theta ) \theta '^2+\frac{1}{2} l^2 m \sin ^2(\theta
   ) \theta '^2+g l m \cos (\theta )+l m \cos (\theta ) x' \theta '
$$
Lo que puede ser reescrito asi:
$$
L=L_0+\frac{1}{2} \textbf{q'}^T T \textbf{q'}
$$
con:
$$
\textbf{q'}=\left(
\begin{array}{l}%[pos]{spalten}
x'\\
\theta'
\end{array} \right)$$ $$
T=
\left(
\begin{array}{ll}
 4 a^2 m x^2+m & l m \cos (\theta )+2 a l m \sin (\theta ) x \\
 l m \cos (\theta )+2 a l m \sin (\theta ) x & l^2 m
\end{array}
\right)$$ $$
L_0=g l m \cos (\theta )-a g m x^2 $$

Para hallar el Hamiltoniano se usa el procedimiento  de la ecuaci\'on 8.27 p\'agina 340 cap\'itulo 8:
$$
H(q,p,t)=\frac{1}{2} \textbf{p}^T T^{-1} \textbf{p}-L_0(q,t)
$$
Para este caso la inversa de $T$ es:
$$
T^{-1}=\frac{1}{\det(T)} \left(
\begin{array}{ll}
 l^2 m & -2 a l m \sin (\theta ) x-l m \cos (\theta ) \\
 -2 a l m \sin (\theta ) x-l m \cos (\theta ) & 4 a^2 m x^2+m
\end{array}
\right)$$ $$
\det(T)=l^2 m^2-l^2 \cos ^2(\theta ) m^2+4 a^2 l^2 x^2 m^2-4 a^2 l^2 \sin ^2(\theta ) x^2
   m^2-4 a l^2 \cos (\theta ) \sin (\theta ) x m^2 $$ $$
\det(T)=m^2 l^2 (\sin(\theta)^2+4 a^2 x^2 \cos^2(\theta)-4 a x \cos (\theta ) \sin (\theta ) )$$ $$
\det(T)=m^2 l^2 (\sin(\theta)-2 a x \cos(\theta))^2
$$

$$
\frac{\textbf{p}^T T^{-1} \textbf{p}}{2}=\frac{1}{2 \det(T)} \left(p_x {} p_\theta \right) \left(
\begin{array}{ll}
 l^2 m & -l m \cos (\theta )-2 a l m \sin (\theta ) x \\
 -l m \cos (\theta )-2 a l m \sin (\theta ) x & 4 a^2 m x^2+m
\end{array}
\right) \left(
\begin{array}{l}
 p_x \\
 p_{\theta }
\end{array}
\right)$$ $$
\frac{\textbf{p}^T T^{-1} \textbf{p}}{2}=\frac{m \left(l^2 p_x^2-2 l p_{\theta } p_x (\cos (\theta )+2 a \sin (\theta ) x)+p_{\theta }^2 \left(4 a^2 x^2+1\right)\right)}{2 \det (T)}
$$
Finalmente el Hamiltoniano queda asi:
$$
H=\frac{m \left(l^2 p_x^2-2 l p_{\theta } p_x (\cos (\theta )+2 a \sin (\theta ) x)+p_{\theta }^2 \left(4 a^2 x^2+1\right)\right)}{2 \det (T)}-m g l \cos (\theta )+m a g x^2$$ $$
H=\frac{l^2 p_x^2-2 l p_{\theta } p_x (\cos (\theta )+2 a \sin (\theta ) x) +p_{\theta }^2 \left(4 a^2 x^2+1\right)}{2 l^2 m (\sin (\theta )-2 a \cos (\theta ) x)^2}-m g l\cos (\theta )+ m g a x^2
$$
Ahora con el hamiltoniano hallamos las ecuaciones de Hamilton:

$$x'=\frac{\partial H}{\partial p_x}=\frac{l p_x-(\cos (\theta )+2 a \sin (\theta ) x) p_{\theta }}{l m (\sin (\theta)-2 a \cos (\theta ) x)^2}$$

$$\theta'=\frac{\partial H}{\partial p_\theta}=\frac{\left(4 a^2 x^2+1\right) p_{\theta }-l (\cos (\theta )+2 a \sin (\theta ) x) p_x}{l^2 m (\sin (\theta )-2 a \cos (\theta ) x)^2}$$

$$p_x'=-\frac{\partial H}{\partial x}=\frac{2 a}{m} \left( -g x m^2+\frac{p_{\theta } \left(l \sin (\theta ) p_x-2 a x p_{\theta   }\right)}{l^2 (\sin \theta -2 a x \cos \theta )^2}  -\frac{\cos (\theta ) \left(l^2 p_x^2-2 l (\cos (\theta )+2 a \sin (\theta ) x) p_{\theta }  p_x+\left(4 a^2 x^2+1\right) p_{\theta }^2\right)} {l^2 (\sin (\theta )-2 a \cos (\theta ) x)^3} \right) $$

\begin{eqnarray*}
p_\theta'&=&-\frac{\partial H}{\partial \theta}= \frac{1}{l^2 m} \left(-g m^2 \sin (\theta ) l^3+\frac{p_x p_{\theta } l}{2 a \cos (\theta ) x-\sin(\theta )} \right)\\
& &+\frac{1}{l^2 m} \left(\frac{(\cos (\theta )+2 a \sin (\theta ) x) \left(l^2 p_x^2-2 l(\cos (\theta )+2 a \sin (\theta ) x) p_{\theta } p_x+\left(4 a^2 x^2+1\right) p_{\theta }^2\right)}{(\sin (\theta )-2 a \cos (\theta ) x)^3} \right)
\end{eqnarray*}

La anteriores expresi\'ones para $p_x'$ y $p_\theta'$ se pueden reorganizar asi:
$$
p_\theta'=\frac{(\cos (\theta )+2 a \sin (\theta ) x) p_x^2}{m (\sin (\theta )-2 a \cos(\theta ) x)^3}$$ $$+\left(\frac{2 a \cos (\theta ) x-\sin (\theta )}{l m (\sin(\theta )-2 a \cos (\theta ) x)^2}-\frac{2 (\cos (\theta )+2 a \sin (\theta )x)^2}{l m (\sin (\theta )-2 a \cos (\theta ) x)^3}\right) p_{\theta }p_x$$ $$+\frac{(\cos (\theta )+2 a \sin (\theta ) x) \left(4 a^2 x^2+1\right)p_{\theta }^2}{l^2 m (\sin (\theta )-2 a \cos (\theta ) x)^3}-g l m \sin (\theta)
$$
$$
p_x'=-\frac{2 a \cos (\theta ) p_x^2}{m (\sin (\theta )-2 a \cos (\theta )
   x)^3}$$ $$+\left(\frac{2 a \sin (\theta )}{l m (\sin (\theta )-2 a \cos (\theta )
   x)^2}+\frac{4 a \cos (\theta ) (\cos (\theta )+2 a \sin (\theta ) x)}{l m (\sin
   (\theta )-2 a \cos (\theta ) x)^3}\right) p_{\theta } p_x+$$ $$\left(-\frac{4 x
   a^2}{l^2 m (\sin (\theta )-2 a \cos (\theta ) x)^2}-\frac{2 \cos (\theta ) \left(4
   a^2 x^2+1\right) a}{l^2 m (\sin (\theta )-2 a \cos (\theta ) x)^3}\right)
   p_{\theta }^2-2 a g m x
$$
\end{document}