\documentclass[letterpaper,12pt]{article}
\usepackage[latin1]{inputenc}
\usepackage[dvips]{graphicx}
\usepackage[spanish]{babel}


\textwidth = 16.5 cm
\textheight = 23.5 cm
\oddsidemargin = 0.0 cm
\evensidemargin = 0.0 cm
\topmargin = 0.0 cm
\headheight = 0.0 cm
\headsep = 0.0 cm 

%opening
\title{Ejercicios del cap\'itulo 1 de\\ \emph{Classical Mechanics} de H. Goldstein}
\author{Nicol\'as Quesada M. \\ {\small \sf Instituto de F\'isica, Universidad de Antioquia}}
\date{}
\begin{document}

\maketitle



\section*{Ejercicio 1.6.}
Supongamos que en alg\'un instante $t$ la part\'icula tiene coordenadas $(x',y')$ y que la tangente del \'angulo que forma su velocidad con el eje $x$ esta dada por $\frac{dy(t)}{dx(t)}$. La recta tangente a la curva estar\'a dada por:
\begin{equation}
(y-y')=\frac{dy(t)}{dx(t)} (x-x').
\end{equation}
Por otro lado seg\'un el problema la velocidad siempre debe apuntar a un punto en el eje $x$, al que llamaremos $f(t)$. Pero este punto no es m\'as que el intercepto con el eje $x$ de la recta anteriormente definida. As\'i entonces debemos tener lo siguiente:
\begin{eqnarray}
-y'=\frac{dy(t)}{dx(t)} (f(t)-x')\\
(f(t)-x)dy+y dx=0
\end{eqnarray}
Que es la ecuaci\'on de una ligadura no-hol\'onoma. Para mostrar que la ligadura es no hol\'onoma se trata de buscar un factor integrante para obtener una diferencial total:
\begin{eqnarray}
dF=f_i(f(t)-x)dy+ f_i y dx+f_i \times 0 dt.\\
\end{eqnarray}
De aqu\'i se lee que:
\begin{eqnarray}
\frac{\partial F}{\partial y}=f_i(f(t)-x) \\
\frac{\partial F}{\partial x}= f_i y\\
\frac{\partial F}{\partial t}=0.
\end{eqnarray}
Haciendo derivadas cruzadas entre $x$ y $t$ 
(y teniendo en cuenta que $x$ y $y$ dependen de $t$ s\'olo impl\'icitamente):
\begin{equation}
\frac{\partial \left(\frac{\partial f_i}{\partial x}\right)}{\partial t}=y \frac{\partial f_i}{\partial t}
=0.\end{equation}
Tomando derivadas cruzadas de $t$ y $y$ nos queda que:
\begin{equation}
 0=\frac{\partial \left(\frac{\partial F}{\partial y}\right)}{\partial t}=f_i \frac{\partial f(t)}{\partial t}=0
\end{equation}
Asi o bien $f_i=0$ o $f(t)=cte$, pero lo anterior va encontra de las hip\'otesis ya que $f(t)$ es arbitrario.


\section*{Ejercicio 1.9 }
Tenemos que el Lagrangiano para una part\'icula en presencia de un campo El\'ectrico 
$\vec E=\nabla \phi -\frac{\partial \vec A}{\partial t}$
y un campo magn\'etico  $\vec B=\nabla \times \vec A$ est\'a dado por 
\begin{equation}
L = {1 \over 2} m \vec{v} \cdot \vec{v}  - q\phi + {q \over c} \vec{v} \cdot \vec{A} .
\end{equation}
Se pregunta que efecto tiene sobre el lagrangiano el cambiar el potencial escalar $\phi$ y el potencial vectorial $\vec A$ 
\begin{eqnarray}
\vec A \longmapsto \vec A+\nabla \psi \\
\phi \longmapsto \phi-\frac{1}{c}\frac{\partial \psi}{\partial t}
\end{eqnarray}
sobre el Lagrangiano y las ecuaciones de movimiento de la part\'icula sobre la que act\'uan $\vec E$ y $\vec B$, siendo $\psi(x,y,z,t)$ una funci\'on arbitraria pero diferenciable.
Si sustituimos en el Lagrangiano y reorganizamos t\'erminos se obtiene lo siguiente:
\begin{eqnarray}
 L' &=&{1 \over 2} m \vec{v} \cdot \vec{v} -q (\phi-\frac{1}{c}\frac{\partial \psi}{\partial t})+\frac{q}{c}\vec v \cdot  (\vec A+\nabla \psi)\\
 L'&=&({1 \over 2} m \vec{v} \cdot \vec{v}-q \phi+\frac{q}{c}\vec v \cdot  \vec A)+q\frac{1}{c}(\frac{\partial \psi}{\partial t}+\vec v \cdot \nabla \psi)
\end{eqnarray}
El t\'ermino dentro del par\'entesis es el lagrangiano original $L$ y el segundo es la diferencial total  de $\psi$ con respecto a $t$:
\begin{eqnarray}
\frac{d \psi}{dt}=\frac{\partial \psi}{\partial x}\frac{dx}{dt}+\frac{\partial \psi}{\partial y}\frac{dy}{dt}+\frac{\partial \psi}{\partial z}\frac{dz}{dt}+\frac{\partial \psi}{\partial t}=\nabla \psi \cdot \vec v+\frac{\partial \psi}{\partial t}
\end{eqnarray}
Asi entonces $L'$ se reescribe como $L$ mas la derivada total con respecto a $t$ de una funci\'on arbitraria $\psi$:
\begin{equation}
L'=L+\frac{d \psi}{dt}
\end{equation}
Pero seg\'un la demostraci\'on 8 Las ecuaciones de Lagrange se siguen satisfaciendo si a un Lagrangiano $L$ se le adiciona la derivada total con respecto al tiempo de una funci\'on arbitraria $\psi$, es decir las ecuaciones de movimiento de la part\'icula no cambian.


\section*{Ejercicio 1.19}
Usando coordenadas esf\'ericas y teniendo en cuenta que el p\'endulo es inextensible tenemos que la velocidad se puede escribir as\'i:
\begin{equation}
\vec v=\frac{d\vec r}{dt}=\frac{dr}{dt}\hat r+r\frac{d\theta}{dt}\hat \theta+r \sin \theta \frac{d\phi}{dt} \hat \phi=r\frac{d\theta}{dt}\hat \theta+r \sin \theta \frac{d\phi}{dt} \hat \phi
\end{equation}
y que el potencial gravitacional se puede escribir como (con $z$ positivo hacia abajo):
\begin{equation}
V=-mgz=-mgr \cos \theta
\end{equation}
Asi el Lagrangiano del sistema es:
\begin{eqnarray}
L=\frac{1}{2} m r^2 \left(\sin ^2(\phi ) \dot \theta^2+\dot \phi^2\right) +mgr \cos (\phi ) \\
\frac{\partial L}{\partial \dot \theta}=m r^2 \dot \theta\\
\frac{\partial L}{\partial \dot \phi}=m r^2 \sin^2\theta \dot \phi\\
\frac{\partial L}{\partial \theta}=m r^2 \sin \theta \cos \theta \dot \phi^2-m g r \sin \theta\\
\frac{\partial L}{\partial \phi}=0
\end{eqnarray}
luego las ecuaciones de movimiento son:
\begin{eqnarray}
m r^2 \sin \theta \cos \theta \dot \phi^2-m g r \sin \theta-m r^2 \frac{d\dot \theta}{dt}=0\\
\frac{d }{dt}\left( m r^2 \sin^2\theta \dot \phi\right)=0\\
\end{eqnarray}


\end{document}